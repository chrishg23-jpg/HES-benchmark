\documentclass[12pt]{article}
\usepackage{amsmath, amssymb, bm}
\usepackage{graphicx}
\usepackage{geometry}
\geometry{margin=1in}

\title{Holographic Entropic Spacetime (HES): A Toy Model for Emergent Curvature}
\author{Chris [Your Last Name] \\ \small{Independent Researcher}}
\date{}

\begin{document}

\maketitle

\begin{abstract}
The nature of spacetime remains one of the deepest puzzles in theoretical physics. While General Relativity describes gravity as curvature in a geometric manifold, and Quantum Mechanics governs the probabilistic behavior of particles, a unified framework that explains how spacetime itself emerges from quantum principles remains elusive.

This paper introduces a toy model based on the Holographic Entropic Spacetime (HES) framework, where curvature arises from entropy gradients in a discrete lattice of microstates. The model simulates how local entropic interactions and global feedback dynamics can generate stable geometric structure without invoking mass or classical fields. We present a reproducible benchmark notebook that visualizes this process, offering a hands-on demonstration of how information structure alone can give rise to curvature. The results support the hypothesis that spacetime may be a macroscopic manifestation of underlying entropic dynamics—and open new avenues for exploring emergent gravity from first principles.
\end{abstract}

\section{Introduction}

The nature of spacetime remains one of the deepest puzzles in theoretical physics. While General Relativity describes gravity as curvature in a geometric manifold, and Quantum Mechanics governs the probabilistic behavior of particles, a unified framework that explains how spacetime itself emerges from quantum principles remains elusive.

Recent advances in quantum information theory suggest that entanglement may play a foundational role in the emergence of spacetime geometry. The idea that “geometry is entanglement” has gained traction—but concrete, reproducible models remain rare.

This paper introduces a toy model based on the Holographic Entropic Spacetime (HES) framework, where curvature arises from entropy gradients in a discrete lattice of microstates. The model simulates how local entropic interactions and global feedback dynamics can generate stable geometric structure without invoking mass or classical fields.

We present a reproducible benchmark notebook that visualizes this process, offering a hands-on demonstration of how information structure alone can give rise to curvature. The results support the hypothesis that spacetime may be a macroscopic manifestation of underlying entropic dynamics—and open new avenues for exploring emergent gravity from first principles.


